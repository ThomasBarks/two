<!doctype html>
<html lang="en">

	<head>
		<meta charset="utf-8">

		<title>SEBASTIAN PORRAS GARULO - ASIGENEMT 2</title>

		<meta name="description" content="My presentation for Asignement 2 Part B">
		<meta name="author" content="Sebastián Porras Garulo">

		<meta name="apple-mobile-web-app-capable" content="yes">
		<meta name="apple-mobile-web-app-status-bar-style" content="white">

		<meta name="viewport" content="width=device-width, initial-scale=1.0, maximum-scale=1.0, user-scalable=no, minimal-ui">

		<link rel="stylesheet" href="css/reveal.css">
		<link rel="stylesheet" href="css/theme/white.css" id="theme">

		<!-- Code syntax highlighting -->
		<link rel="stylesheet" href="lib/css/zenburn.css">

		<!-- Printing and PDF exports -->
		<script>
			var link = document.createElement( 'link' );
			link.rel = 'stylesheet';
			link.type = 'text/css';
			link.href = window.location.search.match( /print-pdf/gi ) ? 'css/print/pdf.css' : 'css/print/paper.css';
			document.getElementsByTagName( 'head' )[0].appendChild( link );
		</script>

		<!--[if lt IE 9]>
		<script src="lib/js/html5shiv.js"></script>
		<![endif]-->
	</head>

	<body>

		<div class="reveal">

			<!-- Any section element inside of this container is displayed as a slide -->
			<div class="slides">
				<section>
					<h1>Asignement 2</h1>
					<h3>Part B</h3>
					<p>
						<small>Created by Sebastián Porras Garulo</small>
					</p>
				</section>

				<!-- Example of nested vertical slides -->
				<section>
					<section>
						<h2>Excercise 1</h2>
					</section>
					<section>
						<h2>Question A</h2>
						<p>The expected value of the prospect is <span class="math inline"><em>E</em><em>V</em> = 0.2 ⋅ 40 + 0.6 ⋅ 50 + 0.2 ⋅ 30 = 44</span> and the utility given the formula <span class="math inline"><em>U</em>(<em>x</em>)=∑<sub><em>i</em></sub><em>p</em><sub><em>i</em></sub><em>u</em><sub><em>i</em></sub></span> is <span class="math inline">0.2 ⋅ 40/10 + 0.6 ⋅ 50/10 + 0.2 ⋅ 30/10 = 4.4</span>.<br />
						The certainty equivalent (CE) is calculated by determining the value of x for which an individual is indifferent of receiving the prospect or a certain amount. In this case, since utility is given by <span class="math inline"><em>U</em>(<em>x</em>)=<em>x</em>/10</span>, the CE is calculated as follows:</p>
					</section>
					<section>
						<p><br /><span class="math display"><em>U</em>(<em>x</em>)=4.4</span><br /> <br /><span class="math display">4.7 = <em>x</em>/10</span><br /> <br /><span class="math display"><em>x</em> = 4.4 ⋅ 10 = 44 = <em>C</em><em>E</em></span><br /></p>
					</section>
					<section>
						<h2>Question B</h2>
						<p>The rank dependent utility calculated using <span class="math inline"><em>U</em>(<em>x</em>)=∑<sub><em>i</em></sub><em>π</em><sub><em>i</em></sub> ⋅ <em>u</em><sub><em>i</em></sub></span> and <span class="math inline"><em>w</em>(<em>p</em>)=<em>p</em><sup>2</sup></span> is as follows: <span class="math inline">0.2<sup>2</sup> ⋅ 4 + (0.8<sup>2</sup> − 0.2<sup>2</sup>)⋅5 + (1<sup>2</sup> − 0.8<sup>2</sup>)⋅3 = 4.24</span>.<br />
						Given that the utility is calculated as u(x)=0.1*x , so the amount for which an individual would be indifferent between the prospect or a certain amount would be <span class="math inline">4.24/0.1 = 42.4 = <em>C</em><em>E</em></span>.<br />
					</section>
				</section>

				<section>
					<section>
						<h2>Excercise 2</h2>
					</section>
					<section>
						<p>Under the expected utility theory, prospect E and prospect F yield the same outcome (750). Prospect G and prospect H both yield an expected utility of 1500. Levy and Levy (2002) reject the S-shaped utility function of the Cumulative Prospect Theory (CPT) because they found a preference for E (71% of the subjects) and a preference for G (76% of the subjects. Under CPT, the S-shaped utility function is concave for gains and convex for losses, meaning that individuals are risk-averse for gains and risk-seeking for losses.<br />
						</p>
					</section>
					<section>
						<p>Applied to the prospects illustrated in the question, since the expected utility is the same, individuals should choose prospects F and H. So, in line with the research of Levy and Levy (2002), we also reject the S-shaped curve based on the results of L&amp;L. However, L&amp;L don’t account for the probability weighting function as Kahneman and Tversky (1992) did. They state in their paper:  <br />
							</p>
						</section>
						<section>
							<p> “All probabilities given in the experiments are relatively large <span class="math inline">(<em>p</em> ≥ 0.25)</span>, hence it is unlikely that subjective probability distortion plays an important role in the decision-making process”. When probability weighting is considered (Wakker, P., 2003), the optimal choices shift from E to F and from G to H, implying that the conclusion of L&amp;L is unfounded.<br />
							</p>
						</section>
						<section>
							<div class="figure">
								<img src="WeightedProbability2.jpg" />
								<p class="caption">Figure 1. As we can see, the low probability is weighted highly, while the tall probabilities are weighted lowly.</p>
							</div>
						</section>
						<section>
							<p> The conclusion of Levy and Levy would have been correct if they had taken into account the probability weighting, but because this wasn’t the case, we do not agree with the conclusion that the S-shaped utility curve is rejected.<br />
							</p>
						</section>
				</section>

				<section>
						<h2>End of the Presentation!</h2>
				</section>

			</div>

		</div>

		<script src="lib/js/head.min.js"></script>
		<script src="js/reveal.js"></script>

		<script>
			// Full list of configuration options available at:
			// https://github.com/hakimel/reveal.js#configuration
			Reveal.initialize({
				controls: true,
				progress: true,
				history: true,
				center: true,
				transition: 'convex', // none/fade/slide/convex/concave/zoom
				// Optional reveal.js plugins
				dependencies: [
					{ src: 'lib/js/classList.js', condition: function() { return !document.body.classList; } },
					{ src: 'plugin/markdown/marked.js', condition: function() { return !!document.querySelector( '[data-markdown]' ); } },
					{ src: 'plugin/markdown/markdown.js', condition: function() { return !!document.querySelector( '[data-markdown]' ); } },
					{ src: 'plugin/highlight/highlight.js', async: true, callback: function() { hljs.initHighlightingOnLoad(); } },
					{ src: 'plugin/zoom-js/zoom.js', async: true },
					{ src: 'plugin/notes/notes.js', async: true }
				]
			});
		</script>

	</body>
</html>